\begin{abstract}

This study examines the impact of software optimizations on the energy consumption of web applications. Using a feature-equivalent micro-blogging application (MiniTwit) implemented in seven programming languages and frameworks: Rust Actix, Python Flask, Ruby Sinatra, JavaScript Express, Java Spring, C\# NetCore, and Go Gorilla. We examine three optimization techniques: jemalloc, Just-In-Time (JIT) compilation, and Profile-Guided Optimization (PGO). We deploy each implementation on Raspberry Pi devices and measure energy consumption under simulated user interaction using a hardware-based measurement method. Our findings show that JIT and PGO generally reduce energy consumption. Using JIT reduced energy consumption by 14.52\% in JavaScript and 6.37\% in Ruby, compared to executions without JIT. For Go, enabling PGO reduced energy consumption by 0.72\%. Using jemalloc yields mixed results, reducing consumption in Ruby by 2.08\% but increasing it in others. We confirm a strong linear correlation between power draw and execution time (Pearson r = 0.86, p < 0.001), and observe that the implementation has a greater impact on energy consumption than optimizations alone. We contribute reproducible experimental methods and data, and highlight the need to consider optimization strategies in both academic energy measurement studies and software development.

\end{abstract}
