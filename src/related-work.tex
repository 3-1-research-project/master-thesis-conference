\documentclass[main.tex]{subfiles}
\begin{document}
\section{Related Work}

A number of studies have explored which programming languages are most energy-efficient \cite{pereira2017energy, Pereira_Couto_Ribeiro_Rua_Cunha_Fernandes_Saraiva_2021, Couto_Pereira_Ribeiro_Rua_Saraiva_2017, Gordillo_Calero_Moraga_García_Fernandes_Abreu_Saraiva_2024}. These studies typically implement a set of programmatic problems, in different languages, and then measure energy consumption \cite[Table 1]{pereira2017energy}. Based on these measurements, the authors produce rankings according to energy consumption. A consistent pattern across these rankings is that compiled languages, such as C and C++, generally outperform higher-level languages like C\# and Python in terms of energy efficiency. 

Most studies found a strong correlation between execution time and energy consumption; faster programs typically consumed less energy. However, exceptions exist \cite[Table 7]{Couto_Pereira_Ribeiro_Rua_Saraiva_2017}. 

The methodology used in the previous papers and its results have recently been questioned. \textcite{Kempen_Kwon_Nguyen_Berger_2024} identified significant inconsistencies in the original rankings, arguing that variations in implementation quality, parallelism, and benchmarking conditions skewed the results. When replicating and refining experiments from earlier studies, they found that the choice of programming language has no significant impact on energy consumption other than execution time \cite{Kempen_Kwon_Nguyen_Berger_2024}

\subsection{Energy Measurement of Larger Applications}

\textcite{Philippot_Anglade_Leboucq_2014} developed a tool to measure the energy consumption of web applications on the client side. By evaluating 500 web applications, they demonstrated that green patterns, such as effective caching, could significantly reduce energy usage.

\textcite{Pfeiffer_Trindade_Meding_Harwick} measured the energy consumption of a micro-blogging web application implemented in several languages and web frameworks. Interestingly, their results diverged from the previously mentioned language rankings \cite{pereira2017energy, Pereira_Couto_Ribeiro_Rua_Cunha_Fernandes_Saraiva_2021, Couto_Pereira_Ribeiro_Rua_Saraiva_2017, Gordillo_Calero_Moraga_García_Fernandes_Abreu_Saraiva_2024}; for instance, JavaScript outperformed Rust.

\subsection{jemalloc, JIT, PGO, and Energy Consumption}

Several papers have studied the impact of specific optimizations on energy consumption. \textcite{Li_Wu_Kavi_Mehta_Yadwadkar_John_2023} and \textcite{Lamprakos_Papadopoulos_Catthoor_Soudris_2022} compared memory allocators and found that jemalloc provided performance benefits over standard malloc.

Regarding JIT, \textcite{Stoico_Dragomir_Lago_2025} examined different Python execution modes and found that enabling JIT could make Python up to 90\% more energy efficient. Similar results were reported in other studies \cite{Ournani_Belgaid_Rouvoy_Rust_Penhoat_2021, Hu_John_2006}, reinforcing the view that JIT can have a considerable impact on energy usage.

To date, no studies have been conducted to investigate the impact of PGO on energy consumption, highlighting a gap in the literature.

\subsection{The Focus of this Paper}

This paper aims to build on these insights by studying how the application of specific configurations related to optimizations, including jemalloc, JIT, and PGO, affects the energy consumption of MiniTwit implementations across different programming languages. We extend the measurement setup and application infrastructure developed in our previous research project \cite{Pfeiffer_Offenberg_Pedersen_Landsgaard_Karlsen}, while adapting and adding to the set of MiniTwit implementations from \textcite{Pfeiffer_Trindade_Meding_Harwick}, which were based on the implementations from the DevOps course at IT University of Copenhagen \cite{devops-course}.
\end{document}
